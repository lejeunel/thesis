%%%%%%%%%%%%%%%%%%%%%%%%%%%%%%%%%%%%%%%%%%%%%%
%
%		Thesis Settings
%		Custom settings
%
%		2011
%
%%%%%%%%%%%%%%%%%%%%%%%%%%%%%%%%%%%%%%%%%%%%%%

% %
% %   Use this file for your own custom packages, command-definitions, etc...
% %
% % 

\usepackage[utf8]{inputenc} % Required for inputting international characters
\usepackage[T1]{fontenc} % Output font encoding for international characters
\usepackage{amssymb, amsmath}
\usepackage{pifont}% http://ctan.org/pkg/pifont
\newcommand{\cmark}{\ding{51}}%
\newcommand{\xmark}{\ding{55}}%

\usepackage{multirow}
\usepackage{makecell}
\usepackage{bm}
\usepackage{tabularx}

\usepackage{hyperref}
\newcommand{\footref}[1]{%
    $^{\ref{#1}}$%
}
\usepackage{csvsimple}
\usepackage{arydshln}
\usepackage{subcaption} % for subfigures
\usepackage{xstring}
\usepackage{algorithm}
\usepackage{algpseudocode}
\renewcommand{\algorithmicrequire}{\textbf{Input:}}
\renewcommand{\algorithmicensure}{\textbf{Output:}}

\usepackage{appendix}
\AtBeginEnvironment{subappendices}{%
\chapter*{Appendix}
\addcontentsline{toc}{chapter}{Appendices}
\counterwithin{figure}{section}
\counterwithin{table}{section}
}

\usepackage{xfrac}
\usepackage[section]{placeins}
%\usepackage{mathpazo} % Use the Palatino font by default
%\usepackage{libertine}
%\usepackage{libertinust1math}
\usepackage[T1]{fontenc}
\usepackage{siunitx}
%\usepackage[backend=bibtex,style=ieee]{biblatex} % Use the bibtex backend with the authoryear citation style (which resembles APA)

% \usepackage[block=ragged,natbib=true]{biblatex}

\usepackage{graphicx}
\usepackage{wrapfig}
%\usepackage{caption}
% \usepackage[caption=false]{subfig}
%\captionsetup[subfigure]{width=0.7\textwidth}
% \graphicspath{{Figures/}{./}} % Specifies where to look for included images

\usepackage{tikz}
\usetikzlibrary{positioning,arrows,fit,backgrounds, patterns}

\usepackage[super]{nth}

%\usepackage[autostyle=true]{csquotes} % Required to generate language-dependent quotes in the bibliography

\usepackage{layout}
\usepackage{diagbox}

\usepackage{tikz}
\usepackage{datapie}
\usepackage[automake]{glossaries-extra}
\makeglossaries

\newcommand{\comment}[1]{{\color{red}{\bf comment: #1}}}

