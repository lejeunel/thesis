\begin{acknowledgements}
\addchaptertocentry{\acknowledgementname} % Add the acknowledgements to the table of contents
A received idea of our time and place says that the products of the human intellect find their originative source in the isolated individuals that ``think''.
This conception was later revisited and given a ``scientific'' formulation in the XIX-th century, within a current of thought commonly referred to as ``materialism''.
Karl Vogt, a German scientist and politician who obtained his medical doctorate at the University of Bern in 1839, wrote in his ``Lectures on man ; his place in creation, and in the history of the earth (1869)'':
\textit{The brain secretes thought as the stomach secretes gastric juice, the liver bile, and the kidneys urine}.
Of course, the latter conception has been, since its first formulation up until today, object of relatively courageous and serious critiques, both within academic and non-academic circles.
However, as Hegel said in his ``Preface to the Philosophy of Right (1817)'': \textit{The owl of Minerva spreads its wings only with the coming of the dusk’}, while the above ``common knowledge'' continues to circulate with phlegm and authority through the pores of the civil society.

While the present space of expression is not appropriate to fully develop the intricacies and implications of the above thesis, let alone explore the numerous debates that it has triggered, my intention was to indicate, through this detour, that the sincere words of gratitude that I'm about to express would be reduced to nothing under the authority of this ``common knowledge''.

\bigskip
My first words of gratitude go to Prof. Dr. Raphael Sznitman, who has offered me the opportunity to work on this difficult but interesting topic.
He has been throughout my thesis a committed and valuable source of inspiration.
While my numerous peregrinations in the world of Machine Learning often showed negative results, he would still allow me to explore numerous approaches while always giving meaningful advises.
Also, his direct implication in the writing of papers was precious.

\bigskip
I would also like to thank Jan Grossrieder, Severin Tobler, Philipp Gerber, Olen Andoni, Serife Kuçur, Pablo Màrquez-Neila, and Thomas Kurrman, as well as other students and researchers of our lab who have provided valuable help in the course of this thesis.
In particular, Jan Grossrieder for his excellent contribution to feature learning, Severin Tobler for his outstanding and professional effort on the annotation software and web-platform.
Many thanks go to Olen Andoni and Philipp Gerber for their valuable contributions on the web-platform.
To Serife and Pablo, many thanks for your relevant theoretical inputs and patience.
Last but not least, my thanks go to Thomas Kurmann who, from his scientific contributions, help in TA sessions, to the challenge of building a computer cluster that allowed all of us to run our experiments smoothly, he has been on so many front a very valuable member of our group.

\bigskip
Also, I wish to thank the University of Bern and the Swiss National Science Foundation for their complementary support in the funding of my thesis.

\bigskip
My last words of gratitude go to my loved ones, mother and sister, who have always been supportive and caring.
Despite the recent and past difficulties that we have encountered, we have somehow managed to conserve an elementary basis of dialogue and transparency, which I have tried to fertilize, despite what my lack of patience might have showed.

\bigskip
Last, I wish to acknowledge the precious support of Tatiana in the recent years.
Her love and patience have played a decisive role in my life, and extended way past the frame of this thesis: To the moon and back, without magnetic shoes.


\end{acknowledgements}

%%% Local Variables:
%%% mode: latex
%%% TeX-master: "../main"
%%% End:
