\resetfigpath{intro}

\chapter{Introduction}
\label{intro}

The medical imaging field has recently witnessed the revolution of \gls{ai}, a broad discipline that is a part of computer science whose aim is to conceptualize human ``intellectual'' tasks, and produce computer programs that can replicate them, thereby allowing to relieve human beings from tedious and repetitive tasks.
From classification to segmentation tasks, \gls{ai} already provides flexible and efficient tools that greatly facilitate the daily practice of clinicians.
In particular, large quantities of \gls{mri} brain scans are now efficiently and accurately processed by \gls{ai} systems that help clinicians in the diagnosis of pathologies \cite{sun19}.
Similarly, surgical instruments can be segmented automatically and accurately in real-time to allow laparoscopic interventions of the abdominal cavity by means of robotic technology, thereby decreasing the invasiveness of traditional methods \cite{davinci}.

\Gls{ml}, a term that denote a sub-category of \gls{ai}, is grounded on the ``learning by example'' paradigm.
Whereas traditional approaches generally consisted in explicitly formalizing and implementing complex tasks as a combination of simpler programmatic instructions,
\gls{ml} rather considers that a flexible system can learn such instructions automatically through repeated experiences.

Concretely, as a young child can effectively derive the concept of car using a few visual examples, and be able to identify an ``unseen'' car in a complex environment with near-perfect accuracy,
state-of-the-art \gls{ml} methods leverage the same principle.
However, no matter how well-engineered they are, they need, in order to replicate this simple task with comparable accuracy, thousands of examples and counter-examples of images \cite{ILSVRC15}.

In this scenario, collecting such quantity of example images and annotating them is tedious and cumbersome.
In practice, the annotation task is performed by numerous individuals in parallel.
The annotation of medical data, however, adds the following practical difficulties.
First, the task can require a specific medical expertise.
Second, assuming such experts are available and willing to participate to this task, their daily obligations imposes a limited time budget.
These two factors largely explain why the medical field does not benefit from the latest and greatest \gls{ml} technologies used in more generic applications \cite{orting19}.


\section{Organization and Contribution of this Thesis}
The organization of this thesis is summarized below per chapter:

\textbf{Chapter 2} gives a detailed description of the proposed problematic.
The datasets that were used in the elaboration and testing of the proposed methods are described.
The requirements that ground our annotation protocol is then explicitated.
Next, we describe our software solutions, namely a cross-platform and multi-device annotation software, and a web platform with a front-end that allows users to upload annotations, and a back-end that runs our segmentations algorithms.

\textbf{Chapter 3} gives a detailed description of the proposed problematic.
First, the datasets that were used in the elaboration and testing of the proposed methods are described.
Second, the requirements that ground our annotation protocol is explicitated.
Third, we present software solutions implemented in the frame of thesis, namely a cross-platform and multi-device annotation software, and a web platform where a front-end allows users to upload annotations, and a back-end that runs our segmentations algorithms.

\textbf{Chapter 4} gives an overview of the broad litterature related to the present thesis,
In particular, we first give a state-of-the-art of computer vision methods related to segmentation of images and volumes using sparse annotations.
Next, we review several categories of \gls{ml} applicable to the same scenario, namely \gls{ssl}, \gls{pu}, and \gls{da}.
Last, we add some essential theoretical elements that intervene in the proposed solutions.

\textbf{Chapter 5} contains our first segmentation method that relies on the Expected Exponential loss, optimized using a gradient boosting classifier. The probabilities of unlabeled samples are estimated using label propagation.

\textbf{Chapter 6} gives a thorough study of feature extraction methods applicable to the present scenario.
We devise a simple evaluation framework based on a \gls{rf} classifier that operates on superpixels.
As baselines, we implement traditional feature extraction methods, such as \gls{bovw}.
Another baseline considers unsupervised deep feature extraction, where we use a \gls{cnn} trained to classify natural images.
We then implement various methods based on Deep Learning that integrate motion priors, as well as priors derived from the user-provided point annotations.
Through extensive experiments, we demonstrate the superiority of the proposed deep feature learning method over the baselines.

\textbf{Chapter 7} develops an innovative framework based on multi-object tracking.
Concretely, we formulate our segmentation problem as a maximum a-posteriori problem, where the random variable to optimize is the objectness of superpixels.
Using deep features extracted to the best method of the previous chapter, we train an ensemble of decision trees using a sampling method applicable to Positive/Unlabeled setup.
Next, simple gaussian kernels provide pairwise similarity measures.
These two models allow to define likelihoods of selecting a superpixel given the user's annotations, and transiting from one frame to the next.
Using the network flow paradigm, we then build a graph that connects superpixels according to spatio-temporal relations.

\textbf{Chapter 8}





%%% Local Variables:
%%% mode: latex
%%% TeX-master: "../../main"
%%% End:
