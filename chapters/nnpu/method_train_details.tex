\subsection{Training details, hyper-parameters, and implementation}
We now describe for each component of our pipeline the hyper-parameters and training details.
The code of our foreground prediction model is written in Python using the PyTorch library \cite{paszke19}.
The tracking module is written in C++.
The code of all components is made publicly available \footnote{\url{https://github.com/lejeunel/ksptrack}}.

\subsubsection{Foreground prediction model}
For all our experiments we use a Convolutional Neural Network based on the U-Net architecture \cite{ronneberger15}, as proposed in \cite{ibtehaz20}.
First, in place of simple convolutional layers, we use ``Inception-like'' blocks \cite{szegedy16} to improve robustness to scale variations.
Second, the shortcut connections are replaced by a serie of $3\times3$ convolutional layers with residual connections.
Last, batch normalization layers \cite{ioffe15} are added after each convolutional layer.

For the proposed self-supervised class-prior estimation (Sec. \ref{sec:nnpu} and \ref{sec:pi_estim}), we divide the training into three phases:

\begin{enumerate}
  \item We perform a pre-training phase so as to make our observations more robust.
    In particular, we initialize the weights of our CNN so as to
    make the training focus more on the foreground, as suggested in \cite{lin17}.
    Concretely, we note that the last layer of our decoder is a sigmoid function $\sigma(x)=\frac{1}{1+e^{-x}}$.
    We set the bias of the preceding convolutional layer to $-\log{\frac{1-\pi_{init}}{\pi_{init}}}$, with $\pi_{init}=0.01$.
    All others parameters are initialized using He's method \cite{he15init}.
    The CNN is then trained for $50$ epochs using constant initial prior $\hat \pi_{0}$ following Alg. \ref{alg:sgdnnpu} with a learning rate set to $10^{-4}$.
    \item We optimize the model and class-prior estimates for maximum $100$ epochs as in Alg. \ref{alg:prior_estim} with a learning rate set to $10^{-5}$.
  \item We train using frame-wise priors given by the previous phase for an additional $100$ epochs with a learning rate of $10^{-5}$.
\end{enumerate}

In all phases, we use the Adam optimizer \cite{kingma14} with weight decay $0.01$.
As data augmentation, we use a random combination of additive gaussian noise, bilateral blur, and gamma constrast.

\subsubsection{Recursive Bayesian Estimation}
For the process, transition, and initial covariance matrices, we use diagonal matrices, i.e.
$Q=\sigma_{Q}\mathbb{I}$, $R=\sigma_{R}\mathbb{I}$, and $S=\sigma_{S}\mathbb{I}$, where $\mathbb{I}$ is the identity matrix.
As the observations are often very noisy, we set an observation variance much larger than the process variance, i.e. $\sigma_{Q}=10$ and $\sigma_{R}=0.05$.
$\sigma_{S}$ is set to $0.03$.
The parameters of the control input are set proportional to $\hat \pi_{0}$, i.e. $u_{0}=0.02 \cdot \hat \pi_{0}$, and $u_{T}=0.4 \cdot \hat \pi_{0}$.
For the observation model (Eq. \ref{eq:observ}), we set $\gamma=2$.
The window length of the frame-wise smoothing filter is set proportional to the number of frames: $L=0.05\cdot N$.
The time-period of our stopping condition is set to $T_{s}=10$, and the threshold on the variance is $\tau=0.007$.

\subsubsection{Tracking and superpixelization}
\label{sec:org4560526}
We define for each user-provided 2D location a circle of radius \(R=0.05 \cdot \max\{W,H\}\) centered on the 2D location, where \(W\) and \(H\) are the width and height of frames, respectively.
In order to reduce the number of edges and alleviate the computational requirement, we also prune ``visiting'' edges when their corresponding foreground probability falls below $0.4$.

So as to make the tracking step tracktable, all sequences are pre-segmented into $\sim 1200$ superpixels using SLIC \cite{achanta12}.
Given that our segmentations will be evaluated at pixel-level, the latter pre-processing step induces an upper-bound in accuracy, which we wish to evalute.
To do so, we use the manual ground truth annotation as follows:
Taking as positive the superpixels that overlap the manual ground truth annotations by more than $0.5$ of their area, and the others as negatives, we obtain the optimal superpixel segmentation of our sequence.
We then compare the latter segmentation with the pixel-wise manual ground truth annotation and report the F1-score in Tab. \ref{tab:sp_errors}.

\begin{table}
\centering
\caption{For each type of sequence, we report the maximum F1-score achievable given the early stage superpixel segmentation.}
\label{tab:sp_errors}
\begin{tabular}{lp{1.8cm}}
\toprule
{} &               F1 \\
\midrule
Brain    &  $0.92 \pm 0.02$ \\
Cochlea  &  $0.92 \pm 0.01$ \\
Slitlamp &  $0.92 \pm 0.02$ \\
Tweezer  &  $0.95 \pm 0.01$ \\
\bottomrule
\end{tabular}
\end{table}


%%% Local Variables:
%%% mode: latex
%%% TeX-master: "../../main"
%%% End:
