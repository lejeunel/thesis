\subsection{Spatio-Temporal Regularization}
\label{sec:tracking}

While our \SSnnPU~method leverages all images and point-wise annotations to train and segment the data volume in question, the output of our method does not explicitly leverage the spatio-temporal relations within the data cube. That is, every sample is treated and predicted independently, and only implicitly related through $f_\theta$. In order to coherently regularize over the different frames and locations, we use an existing graph based framework as a post-processing step.

To this end, we make use of the multi-object tracking framework (\KSPTrack) introduced in~\cite{lejeune18} to refine the output of the \SSnnPU~method. In short, \KSPTrack~represents the data volume with superpixels and builds a network graph over these to optimize a set of spatio-temporal paths that jointly correspond to the object segmentation throughout the data volume. This is solved by casting the problem as network-flow optimization, whereby costs are assigned to input/output nodes, visiting and transition edges within and across frames, and where 2D annotations are used to define source nodes that allow to push flow within the network.

In practice, we use the same orginal \KSPTrack~setup as in~\cite{lejeune18} with the exception of using the output of \SSnnPU, $f_{\theta}(x_i)$ to compute the cost of selecting superpixel $x_i$ as part of the object by,
\begin{equation}
  \label{eq:cost_fg}
  C_{fg}(i) = -\log \frac{f_\theta(x_i)}{1-f_\theta(x_i)},
\end{equation}
\noindent
where $C_{fg}(i)$ is the cost of including superpixed $x_i$ as part of the object. By construction, this relation therefore imposes a negative cost when $f_{\theta}(x) > 0.5$, and a non-negative cost otherwise. 

The final output of the \KSPTrack~method yields a binary image for each of the frames in the data volume. For the remainder of this paper, we will refer to the combined use of \SSnnPU~and \KSPTrack~as \SSnnPUKSP.

%%% Local Variables:
%%% mode: latex
%%% TeX-master: "main"
%%% End:
