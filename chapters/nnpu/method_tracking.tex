\subsection{Tracking Framework}
\label{sec:tracking}
In practice, the foreground predictions, as given by the output of our model, are noisy.
As a refinement step, we resort to a similar strategy as \cite{lejeune18}, where the spatio-temporal relations of superpixels are leveraged in a multi-object tracking framework.
We now briefly describe the latter approach, and show where the proposed foreground prediction method fits.


\subsubsection{Tracking as a Linear Program}
\label{sec:orgcf5c794}
Let \(\mathcal{P}=\{x_i\}_{i=1}^{|\mathcal{P}|}\) as a path formed by temporally-ordered superpixels \(x_{i}\), where we omit the time index for conciseness.
The segmentation task reduces to finding an optimal set of paths \(\bm{\mathcal{P}}^*=\{ \mathcal{P}_k\}_{k=1}^K\) so as to minimize a cost.
As further developped in \cite{lejeune18}, the latter problem is first formulated as a maximum a posteriori problem and converted to the integer program:

\begin{equation}
\label{eq:lin_prog}
\bm{\mathcal{P}}^* = \operatorname*{argmin}_{\bm{\mathcal{P}}} \sum_i C_{fg}(i) f_{visit}(i) + \sum_{i,j} C_{trans}(i,j)f_{trans}(i,j) + \sum_{i} C_{in} f_{in}(i)
\end{equation}

Considered as a network flow problem, the latter integer program is further restricted to edge capacity constraints and flow conservation at the nodes inputs and outputs.

The term $C_{fg}(i)$ defines the cost of selecting superpixel $x_i$ as foreground.
We derive it from the foreground probability of the corresponding superpixel as

\begin{equation}
  \label{eq:cost_fg}
  C_{fg}(i) = -\log \frac{f_\theta(x_i)}{1-f_\theta(x_i)}
\end{equation}

By construction, this relation therefore imposes a negative cost when $f_{\theta}(x) > 0.5$, and a non-negative cost otherwise.
The costs $C_{in}(i)$, express the cost of pushing flow into the network starting from superpixel $x_{i}$.
In particular, for each user-provided 2D location (assumed to be located on the object), we define a circle centered on it of radius $R$.
$C_{in}(i)$ is set to $0$ when the centroid of superpixel $x_i$ is contained within that circle, and $\infty$ otherwise.
The transitions costs $C_{trans}(i,j)$, which define the cost of transiting from $x_i$ to $x_j$, are set to $0$ when $x_i$ and $x_j$ overlap, i.e. have at least $1$ pixel of same location, and $\infty$ otherwise.

To optimize the problem of Eq. \ref{eq:lin_prog}, we resort to the K-Shortest Paths (KSP) algorithm as described in \cite{lejeune18}.

Note that in contrast with \cite{lejeune18}, we simplify the modeling of costs $C_{in}$ and $C_{trans}$.
Also, the latter work performs several iterations of the tracking algorithm by augmenting the set of positives and re-training their foreground model after each iteration.
This augmentation strategy did not prove beneficial in the proposed foreground prediction model.


%%% Local Variables:
%%% mode: latex
%%% TeX-master: "main"
%%% End:
