\section{Methods}
\label{sec:methods}

The goal of our method is to generate a segmentation mask for an object of interest in each frame of a single image volume or video sequence using only point-wise annotations and without knowing the image type or object of interest beforehand.

To do so, we propose a novel approach within the Positive-Unlabeled learning setting, that learns the segmentation of the object identified by the point-wise annotations. Our method makes use of a non-negative risk estimator~\cite{kiryo17}, which heavily relies on knowing the proportion of positive samples in the data. While unavailable in our setting, we introduce a novel self-supervised method to estimate this key value via an iterative learning procedure within a Bayesian estimation framework. We then devise a stopping condition to halt training at an appropriate point. Last, a spatio-temporal tracking framework is applied to regularize the output of our approach over the complete volume information.

We describe our approach now in more detail. In the following subsection, we introduce the Positive-Unlabeled learning framework and its non-negative risk estimator. We then describe in 
Sec.~\ref{sec:pi_estim} our self-supervised approach to learn effective class priors. Last, we detail how we leverage the spatio-temporal regularizer, and provide our implementation details
in Sec.~\ref{sec:tracking} and Sec.~\ref{sec:implementation}, respectively.

\subsection{Non-negative Positive-Unlabeled learning}
\label{sec:nnpu}

In Positive/Negative learning (PN) applied to image foreground prediction task,
one usually defines a model \(f_{\theta}: I \mapsto [0;1]^{W \cdot H}\), where \(\theta\) is a set of parameters, \(I\) is the input image space, and $(W,H)$ are its width and height.

Let \(\bm{\mathcal{I}} = \{\mathcal{I}^i\}_{i=1}^{N}\) a set of $N$ input images.
For practical reasons, the present work considers that annotations are provided at a superpixel-level.
Note however that the following development remains valid when annotations are provided at pixel-level.
Each image $\mathcal{I}^i$ is therefore composed of annotated superpixels, $\mathcal{X}^i=\mathcal{X}_p^i \cup \mathcal{X}_n^i$, where $\mathcal{X}_p^i$ and $\mathcal{X}_n^i$ denote the positive and negative superpixels, respectively.
To simply notations we write $f_{\theta}(x) = f_{\theta}(\mathcal{I})\big|_x$ as the
response of $f_{\theta}$ on image $\mathcal{I}$ averaged on the region defined by superpixel $x$.

Given a known class prior for every image $\bm{\pi} = \{\pi^i\}_{i=1}^{N}$, one aims to optimize $\theta$ by minimizing an objective of the form:
\begin{multline}
  \label{eq:pn}
R_{pn}=\sum_{i=1}^{N} \left[ \frac{\pi^i}{|\mathcal{X}_p^i|}\sum_{x \in \mathcal{X}_p^i}\ell(f_{\theta}(x),+1) + \\
\frac{1-\pi^i}{|\mathcal{X}_n|}\sum_{x \in \mathcal{X}_n^i}\ell(f_{\theta}(x),-1) \right]
\end{multline}

A popular objective is the logistic loss, for which \(\ell(z,+1)=\log(1+ e^{-z})\), \(\ell(z, -1)=\log(1+e^{z})\) are the positive and negative entropy loss terms, respectively.
When using both the logistic loss and class-priors, the objective of Eq. \ref{eq:pn} is called Balanced Cross-Entropy loss (BBCE).

In a PU setup, the objective of Eq. \ref{eq:pn} is inadequate as we do not have $\bm{\mathcal{X}}_{n}$.
Instead, we have a set of unlabeled superpixels $\bm{\mathcal{X}}_{u}$ which can contain both positives and negatives.
As suggested in \cite{duplessis15}, the negative risk, i.e. the second term of Eq. \ref{eq:pn} can be derived as follows.
Let \(p(x|Y=1)\), \(p(x|Y=-1)\), and $p(x)$ the probability distributions of positive, negative, and unlabeled samples, respectively.
Without loss of generality, we take $\pi_{i}=\pi, \forall i$.
The distribution of unlabeled samples can be written:
\begin{equation}
  \label{eq:probas}
   p(x) = \pi p(x|Y=1) + (1-\pi) p(x|Y=-1)
\end{equation}

We then express the estimated negative risk as:
\begin{multline}
  \label{eq:estim_neg_risk}
   (1-\pi) \mathbb{E}_{X \sim p(x|Y=-1)}\left[\ell(f_\theta(X,-1)) \right] = \\
    \mathbb{E}_{X \sim p(x)}\left[\ell(f_\theta(X,-1)) \right]
    - \pi \mathbb{E}_{X \sim p(x|Y=+1)}\left[\ell(f_\theta(X,-1)) \right]
\end{multline}

The two bottom terms of Eq. \ref{eq:estim_neg_risk} can be approximated using the available data to give the empirical risk.
Injecting the latter into Eq. \ref{eq:pn}, one gets the empirical PU risk estimator:
\begin{multline}
  \label{eq:nnpu}
R_{pu}=\sum_{i=1}^{N}\Biggl[ \frac{\pi^{i}}{|\mathcal{X}^{i}_{p}|}\sum_{x \in \mathcal{X}^{i}_p}\ell(f_{\theta}(x),+1) + \\
\Biggl( \frac{1}{|\mathcal{X}^{i}_{u}|}\sum_{x \in \mathcal{X}^{i}_u}\ell(f_{\theta}(x),-1) -
\frac{\pi^{i}}{|\mathcal{X}^{i}_{p}|}\sum_{x \in \mathcal{X}^{i}_p}\ell(f_{\theta}(x),-1) \Biggr) \Biggr]
\end{multline}

In practice, we aim at minimizing the objective of Eq. \ref{eq:nnpu} using a Convolutional Neural Network by applying stochastic gradient descent on mini-batches of images.
However, such expressive models tend to overfit to the training data, which tends to make the negative risk, i.e. the bottom term of Eq. \ref{eq:nnpu} go negative.
To circumvent that, \cite{kiryo17} suggest two solutions.
(1) Clip the negative risk to be non-negative, or (2) perform gradient ascent when the negative risk the mini-batch is negative.
Preliminary experiments showed that the second option works best.

In particular, the negative risk writes
\begin{multline}
  \label{eq:neg_risk}
R_{i}^{-}=\sum_{i}\Biggl(
 \frac{1}{|\mathcal{X}^{i}_{u}|}\sum_{x \in \mathcal{X}^{i}_u}\ell(f_{\theta}(x),-1) - \\
\frac{\pi^{i}}{|\mathcal{X}^{i}_{p}|}\sum_{x \in \mathcal{X}^{i}_p}\ell(f_{\theta}(x),-1) \Biggr)
\end{multline}

When \(R_{i}^{-} < 0\), we do gradient ascent by setting the gradient to \(-\nabla_\theta R_{i}^{-}\).
When \(R_{i}^{-} \geq 0\), the gradient is set to \(\nabla_\theta R_{pu}\).
Our procedure is described in Alg. \ref{alg:sgdnnpu}.

\algnewcommand\algorithmicforeach{\textbf{for each}}
\algdef{S}[FOR]{ForEach}[1]{\algorithmicforeach\ #1\ \algorithmicdo}
\begin{algorithm}[H]
\caption{Non-negative PU learning}
\label{alg:sgdnnpu}
\begin{algorithmic}[1]
\Require{$f_\theta$: Prediction model \newline
  $\bm{\mathcal{I}}$: Set of images \newline
  $\bm{\mathcal{X}_p}, \bm{\mathcal{X}_u}$: Positive and unlabeled superpixels \newline
  $\bm{\pi}$: Set of class priors \newline
  $T$: Number of epochs \newline}

\For{\texttt{epoch} $\gets 1$ to $T$}
\State Shuffle dataset into $N_{b}$ batches
    \For{$i \gets 1$ to $N_{b}$}
      \State Sample next batch to get $\mathcal{I}^i$, $\mathcal{X}_p^i$, $\mathcal{X}_u^i$, $\bm{\pi}^i$
      \State Forward pass $\mathcal{I}^i$ in $f_\theta$
      \State Compute risks as in Eq. \ref{eq:nnpu} and \ref{eq:neg_risk}
      \If{$R_{i}^{-} < 0$}
          \State Do gradient ascent along $\nabla_\theta R_{i}^{-}$
      \Else
          \State Do gradient descent along $\nabla_\theta R_{pu}$
      \EndIf
  \EndFor
\EndFor
\end{algorithmic}
\end{algorithm}


%%% Local Variables:
%%% mode: latex
%%% TeX-master: "../../main"
%%% End:

\subsection{Self-supervised Class-priors Estimation}
\label{sec:pi_estim}

As will be shown in our experiments, the optimization problem of Eq. \ref{eq:nnpu} requires frame-wise class-priors $\bm{\pi}$ to be effective.
As the latter parameter is unknown in the present scenario,
we devise a self-supervised strategy to estimate it from data.

In a nutshell, our approach works as follows.
We aim at finding for all frames an estimate of the true class-prior by iteratively training a prediction model with decreasing priors.
As initial priors, we consider an upper-bound.
After having trained our prediction model using the initial prior, we use the predictions of the same model as evidences to decrease the previous priors in a recursive bayesian fashion.
The latter two steps are repeated until a stopping condition is verified.

We first formalize our problem as a state-space model.
Next, we describe how the latter can be solved using the recursive bayesian estimation paradigm.
Last, we suggest a stopping condition.

\subsubsection{State-space model}

Let $\bm{\tilde\pi}_k=\{\tilde\pi_{k}^{i}\}_{i=1}^N$,
the hidden state variable representing the class prior at time $k$.
Furthermore, let $\hat \pi_0 > \pi^{i}, \forall i$ an initial upper-bound on the true prior.
We consider a model that provides \(\bm{\rho}_k=\{\rho_{k}^i\}_{i=1}^N\), a noisy observation of $\bm{\tilde\pi}_{k}$.
Our idea is to progressively decrease $\bm{\tilde\pi}_{k}$ from its initial value $\hat \pi_{0}$ by using $\bm{\rho}_{k}$ as observations.
We introduce the following state-space model:

\begin{align}
\bm{\bm{\tilde\pi}}_{k+1} &= g(\bm{\tilde\pi}_{k}, L) - u_{k}\mathbf{1}_{N} + \mathcal{N}(0,Q), \quad \tilde\pi_{k}^{i} \in [0; \pi_{{max}}] \label{eq:trans_fn}\\
\bm{\bm{\rho}}_{k} &= \bm{\tilde\pi}_{k} + \mathcal{N}(0,R) \label{eq:proc_fn} \\
\bm{\tilde\pi}_{0}&=\hat \pi_{0} + \mathcal{N}(0,S) \label{eq:init_fn}
\end{align}

Where $Q$, $R$, and $S$ are the transition, observation, and initial covariance matrices, respectively.
The function $g(., L)$ is a moving average filter of length $L$ with a Hanning window which imposes a frame-wise smoothness.
For convenience, we write $\mathbf{1}_{N}$ for a vector of length $N$ taking values of $1$.
The term $u_{k}$ is the control input:
\begin{equation}
u_{k} = u_{0} + (u_{T} - u_{0})\frac{k}{T}
\end{equation}

Where $u_{0}$ and $u_{T}$ two constants such that $u_{T} > u_{0}$.
This term therefore induces a downward acceleration on the states and imposes a ``sweeping'' effect on the latter, which allows in principle the range $[0; \hat \pi_{0}]$ to be explored.

\subsubsection{Observation model}
\label{sec:obs_model}
We are now interested in inferring robust obervations from our prediction model.
For clarity, we write $f_{\theta_{k}}$ for our prediction model trained for $k$ epochs.
Let $f_{\theta_{k}}(x)$ the prediction given by $f_{\theta_{k}}$ on superpixel $x$ at iteration $k$.
We therefore compute $\rho_{k}^{i}$, the observation at time $k$ on frame $i$ as:

\begin{equation}
  \label{eq:observ}
\rho_{k}^{i} = \mathbb{E}_{x \in \mathcal{X}^{i}}[f_{\theta_{k}}(x)^{\gamma}]
\end{equation}

With $\gamma > 1$ a correction factor.
This correction is justified as follows: As early iterations use class-priors that are above the true values,
our prediction model tends to over-segment the object of interest and therefore over-estimate the frequencies of positives.

\subsubsection{Recursive Bayesian Estimation}
Our ultimate goal is to obtain an optimal estimate of our state.
As criterion of optimality, we choose the mean-squared error.
The optimal state estimate is therefore given by the conditional mean:

\begin{equation}
  \label{eq:cond_mean}
\bm{\hat\pi}_{k} = \mathbb{E}[\bm{\tilde \pi}_{k} | \bm{\rho}_{0:k}]
\end{equation}

Where $\bm{\hat\pi}_k$ is the optimal estimate of $\bm{\tilde\pi}_{k}$ given $\bm{\rho}_{0:k}$, the sequence of observations from time $0$ to $k$.
The latter conditional expectation requires the knowledge of the a posteriori probability density function (PDF) $p(\bm{\tilde\pi}_{k}|\bm{\rho}_{0:k})$.
Using Bayes' rule, we write:
\begin{equation}
  \label{eq:aposteriori}
  p(\bm{\tilde\pi}_{k}|\bm{\rho}_{0:k}) = \frac{p(\bm{\tilde\pi}_{k}|\bm{\rho}_{0:k-1})\cdot p(\bm{\rho}_{k}|\bm{\tilde\pi}_{k})}{p(\bm{\rho}_{k}|\bm{\rho}_{0:k-1})}
\end{equation}

Where
\begin{equation}
  \label{eq:prior}
  p(\bm{\tilde\pi}_{k}|\bm{\rho}_{0:k-1}) = \int p(\bm{\tilde\pi}_{k}|\bm{\tilde\pi}_{k-1}) \cdot p(\bm{\tilde\pi}_{k-1}|\bm{\rho}_{0:k-1}) d\bm{\tilde\pi}_{k-1}
\end{equation}

is a recursive expression of the state PDF at time $k$ as a function of the state PDF at time $k-1$, and the most recent observations.
The bottom term of Eq. \ref{eq:aposteriori} is a normalization factor that writes:

\begin{equation}
  \label{eq:norm_cst}
  p(\bm{\rho}_{k}|\bm{\rho}_{0:k-1}) = \int p(\bm{\tilde\pi}_{k}|\bm{\rho}_{0:k-1}) \cdot p(\bm{\rho}_{k}|\bm{\tilde\pi}_{k}) d\bm{\tilde\pi}_{k}
\end{equation}

Our state-space model provides expressions of the state-transition probability $p(\bm{\tilde\pi}_{k}|\bm{\tilde\pi}_{k-1})$ via Eq. \ref{eq:trans_fn}, and likelihood $p(\bm{\rho}_{k}|\bm{\tilde\pi}_{k})$ via Eq. \ref{eq:proc_fn}.

In a recursive bayesian filtering application, one distinguishes two phases: prediction and update, where the prediction phase computes the a-priori state density (Eq. \ref{eq:prior}) using the transition function.
In the update phase, a new observation vector is available that allows to compute the likelihood $p(\bm{\rho}_{k}|\bm{\tilde\pi}_{k})$ and normalization constant of Eq. \ref{eq:norm_cst}.
Last, the a-posteriori state estimate is computed using Eq. \ref{eq:aposteriori}.

Modeling the states as a multi-variate gaussian random variable (GRV) with additive gaussian noise greatly simplifies the computation of Eq. \ref{eq:prior} and \ref{eq:norm_cst}.
Assuming further that the transition and observation models are linear, one typically resort to the trusted Kalman Filter (KF) \cite{kalman1960}.

However, the present scenario imposes an inequality constraint on the states so as to make them interpretable as probabilities, a requirement that standard KF does not handle.
In \cite{gupta07}, authors suggest to apply an intermediate step in which the a-priori state estimates are projected on the constraint surface.
This approach, despite being effective, requires the solving of a quadratic program at each iteration.

Our solution follows the simpler approach of \cite{kandepu08}, who relie on the Unscented Kalman Filter (UKF) approach \cite{wan00}.
In contrast with standard KF, which propagate the means and covariances of states through the (linear) system, UKF samples a set of carefully chosen points from the state distribution, called sigma-points, that allow to accurately approximate the true statistics of the posterior.
Our inequality constraints are then directly applied to the sigma-points.
% TODO: introduce capital pi

\subsubsection{Stopping condition}
By construction, our state-space model tends to force the states to decrease, and often to go past below the true values.
We therefore devise a composite stopping condition.

Let $\bm{\tilde{\mathcal{X}}}_{n}=\{x \in \bm{\mathcal{X}} | f_{\theta}(x) < 0.5\}$, and $\bm{\tilde{\mathcal{X}}}_{p}=\{x \in \bm{\mathcal{X}} | f_{\theta}(x) \geq 0.5\}$ the set of ``pseudo-negative'' and ``pseudo-positive'' superpixels, respectively.
As a first criteria, we use the variance of the predictions of  $\bm{\tilde{\mathcal{X}}}_{n}$, written $Var[f_{\theta}(\bm{\mathcal{\tilde X}}_{n})]$, which gives a measure of the confidence of our model on the background.
Second, we also impose that our predictions are such that the frequency of positives is below our upper-bound on all frames.
Concretely, our composite stopping criteria are:

\begin{enumerate}
\item Impose a maximum on the frequencies of pseudo-positives, i.e. $\frac{|\bm{\tilde \mathcal{X}}_{p}^{i}|}{N_{i}} < \hat \pi_{0} \quad \forall i$.
\item Impose a maximum variance level to pseudo-negatives, i.e. $Var[f_{\theta}(\bm{\mathcal{\tilde X}}_{n})] < \tau$.
\end{enumerate}

In practice, we want the above conditions to be verified for several iterations so as to guarantee stability.
We therefore impose that these conditions are verified for $T_{s}$ iterations.
In particular, we denote $C(\bm{\hat \pi}_{k})$ a boolean function that represents the above conditions, and select the optimal prior $\bm{\hat \pi}^{*}$ as

\begin{equation}
  \label{eq:prior_opt}
  \bm{\hat \pi}^{*} = \bm{\hat \pi}_{k} \quad \text{iff} \quad C(\bm{\hat \pi}_{k}) \wedge
  C(\bm{\hat \pi}_{k+1})
  \wedge \cdots \wedge  C(\bm{\hat \pi}_{k+T_{s}})
\end{equation}


\begin{algorithm}[H]
\caption{Self-supervised class-prior estimation}
\label{alg:prior_estim}
\begin{algorithmic}
\Require{$\hat \pi_{0}$: Upper-bound on class-priors} \newline
  $T$: Number of epochs \newline
  $f_\theta$: Foreground prediction model
\Ensure {Optimal estimate of class-prior $\bm{\hat{\pi}}^{*}$}
\State $k \gets 0$
\While {Stopping condition is not verified}
    \State Optimize $f_\theta$ for $1$ epoch as in Alg. \ref{alg:sgdnnpu} using priors $\bm{\hat{\pi}}_{k}$
    \State Forward pass all images to get $\bm{y}_k$
    \State Compute observations $\bm{\rho}_k$ from $\bm{y}_{k}$ as in Eq. \ref{eq:observ}
    \State Clip $\bm{\rho}_{k}$ to $[0,\hat \pi_{0}]$
    \State Denote $\bm{\hat\Pi}_{k-1}$ the sigma-points of $\bm{\hat\pi}_{k-1}$
    \State Clip $\bm{\hat\Pi}_{k}$ to $[0, \hat \pi_{0}]$
    \State Transform $\bm{\hat\Pi}$ through the state-transition function to get $\bm{\hat\Pi}_{k+1}^{-}$
    \State Clip $\bm{\hat\Pi}_{k+1}^{-}$ to $[0, \hat \pi_{0}]$
    \State Compute a-posteriori state $\bm{\hat\pi}_{k+1}$ using observations $\bm{\rho}_{k}$
    \State $k \gets k+1$
\EndWhile
\end{algorithmic}
\end{algorithm}

Alg. \ref{alg:prior_estim} summarizes our approach.
On Fig. \ref{fig:prevs_conv}, we illustrate its behaviour by showing the predicted probabilities and their corresponding class-priors for a Brain sequence.

\begin{figure*}[t]
\caption{Visual example of our self-supervised class-prior estimation. (Top row): Original image with groundtruth highlighted in red and 2D location in green, output prediction at different epochs. (Bottom row): Priors at corresponding epochs. The true priors are in blue, observations are in orange, frequencies of classifier are in green, and current state estimate is in red.}
\centering
    \includegraphics[width=.9\textwidth]{pics/prevs_conv.png}
\label{fig:prevs_conv}
\end{figure*}


%%% Local Variables:
%%% mode: latex
%%% TeX-master: "main"
%%% End:

\subsection{Spatio-Temporal Regularization}
\label{sec:tracking}

While our \SSnnPU~method leverages all images and point-wise annotations to train and segment the data volume in question, the output of our method does not explicitly leverage the spatio-temporal relations within the data cube. That is, every sample is treated and predicted independently, and only implicitly related through $f_\theta$. In order to coherently regularize over the different frames and locations, we use an existing graph based framework as a post-processing step.

To this end, we make use of the multi-object tracking framework (\KSPTrack) introduced in~\cite{lejeune18} to refine the output of the \SSnnPU~method. In short, \KSPTrack~represents the data volume with superpixels and builds a network graph over these to optimize a set of spatio-temporal paths that jointly correspond to the object segmentation throughout the data volume. This is solved by casting the problem as network-flow optimization, whereby costs are assigned to input/output nodes, visiting and transition edges within and across frames, and where 2D annotations are used to define source nodes that allow to push flow within the network.

In practice, we use the same orginal \KSPTrack~setup as in~\cite{lejeune18} with the exception of using the output of \SSnnPU, $f_{\theta}(x_i)$ to compute the cost of selecting superpixel $x_i$ as part of the object by,
\begin{equation}
  \label{eq:cost_fg}
  C_{fg}(i) = -\log \frac{f_\theta(x_i)}{1-f_\theta(x_i)},
\end{equation}
\noindent
where $C_{fg}(i)$ is the cost of including superpixed $x_i$ as part of the object. By construction, this relation therefore imposes a negative cost when $f_{\theta}(x) > 0.5$, and a non-negative cost otherwise. 

The final output of the \KSPTrack~method yields a binary image for each of the frames in the data volume. For the remainder of this paper, we will refer to the combined use of \SSnnPU~and \KSPTrack~as \SSnnPUKSP.

%%% Local Variables:
%%% mode: latex
%%% TeX-master: "main"
%%% End:

\subsection{Training details, hyper-parameters, and implementation}
\label{sec:implementation}
We now specify technical details of our implementation and training procedure. \SSnnPU~is implemented\footnote{Code will be made publicly available.} in Python using the PyTorch library, while we use the publicly available implementation of \KSPTrack\footnote{\url{https://github.com/lejeunel/ksptrack}}.

\subsubsection{SSnnPU}
$f_\theta$ is implemented as a Convolutional Neural Network based on the U-Net architecture proposed in~\cite{ibtehaz20} for all experiments. It uses ``Inception-like'' blocks in place of simple convolutional layers to improve robustness to scale variations. Skip connections are replaced by a serie of $3\times3$ convolutional layers with residual connections. Batch normalization layers are added after each convolutional layer.

To train \SSnnPU, we proceed with a three phase process:
\begin{enumerate}
  \item To increase the robustness of early observations, we train $f$ for 50 epochs with Alg.~\ref{alg:sgdnnpu} and a learning rate set to $10^{-4}$. With the last layer of our decoder being a sigmoid function, we set the bias of the preceding convolutional layer to $-\log{\frac{1-\pi_{init}}{\pi_{init}}}$, with $\pi_{init}=0.01$, as advised in~\cite{lin17}. All others parameters are initialized using He's method~\cite{he15init}.
    \item We then optimize the model and class-prior estimates for a maximum $100$ epochs as described in Alg.~ \ref{alg:prior_estim} with a learning rate set to $10^{-5}$.
  \item We then train using frame-wise priors given by the previous phase for an additional $100$ epochs with a learning rate of $10^{-5}$.
\end{enumerate}
We use the Adam optimizer with weight decay $0.01$ for all training phases. Data augmentation is performed using a random combination of additive gaussian noise, bilateral blur and gamma contrasting. 

\subsubsection{Recursive Bayesian Estimation}
For the process, transition, and initial covariance matrices, we use diagonal matrices 
$Q=\sigma_{Q}\mathbb{I}$, $R=\sigma_{R}\mathbb{I}$, and $S=\sigma_{S}\mathbb{I}$, where $\mathbb{I}$ is the identity matrix.
As the observations $\rho_{k}^{i}$ are often very noisy, we set $\gamma=2$ and the observation variance much larger than the process variance $\sigma_{Q}=10$, $\sigma_{R}=0.05$ and $\sigma_{S}=0.03$.
The parameters of the control input are set proportionally to $\hat \pi_{0}$ with $u_{0}=0.02 \hat \pi_{0}$, and $u_{T}=0.4 \hat \pi_{0}$. The window length of the frame-wise smoothing filter is set proportionally to the number of frames: $L=0.05N$. The time-period of our stopping condition is set to $T_{s}=10$ and the threshold on the variance is $\tau=0.007$.

\subsubsection{KSPTrack parameters}
\label{sec:org4560526}
All sequences are pre-segmented into $\sim 1200$ superpixels and the output of $f$ is averaged over all pixels in a superpixel. Each point-wise annotation defines a circle of radius \(R=0.05 \cdot \max\{W,H\}\) centered on the 2D location, where \(W\) and \(H\) are the width and height of frames, respectively. The input cost at given superpixel is set to $0$ when its centroid is contained within that circle, and $\infty$ otherwise. The transitions costs are set to $0$ when superpixels overlap and $\infty$ otherwise.
In order to reduce the number of edges and alleviate the computational requirements, we also prune {visiting} edges when their corresponding object probability falls below $0.4$. We perform a single round of \KSPTrack~as augmenting the set of positives and re-training the object model after each round (as in~\cite{lejeune18}) did not prove beneficial.

%%% Local Variables:
%%% mode: latex
%%% TeX-master: "main"
%%% End:



%%% Local Variables:
%%% mode: latex
%%% TeX-master: "main"
%%% End:
