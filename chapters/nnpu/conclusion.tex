\section{Conclusions}
\label{sec:conclusion}

The present work contributes to the challenging problem of segmenting medical sequences of various modalities using
point-wise annotations.
From the annotator's point of view, the burden is marginally increased in that we require, aside from a maximum of one 2D location per frame, an upper-bound on the class-prior.

By formulating our problem as a positive/unlabeled prediction task, we demonstrated the relevance of the non-negative unbiased risk estimator in the latter scenario by applying it as the loss function of a Deep Convolutional Neural Network.
We further contributed by devising a self-supervised framework based on recursive bayesian filtering, to estimate the class priors, a hyper-parameter that plays an important role in the segmentation accuracy.
We also combined the latter foreground prediction method with a tracking framework that explicitly leverages the temporal relations of over-segmented regions.

We demonstrated through extensive experiments substantial improvements over the state-of-the-art.
The proposed self-supervised estimation method is proven to be resilient to misspecification of the prior's upper-bound.

%%% Local Variables:
%%% mode: latex
%%% TeX-master: "main"
%%% End:
