\chapter{Deep Feature Learning}
This chapter investigates several feature extraction methods as a contribution to the sparse point-wise annotation of the next chapter.

\textbf{Author contributions} The software solutions presented in this chapter combine the contributions of Severin Tobler, Jan Grossrieder, Philipp Gerber, Olen Andoni, and myself.
Severin Tobler, as part of his civil service, contributed to the coding of the annotation software, in particular, the design of the \gls{gui}, coding the video/volume decoding layers, and the documentation.
Jan Grossrieder, Philipp Gerber, and Olen Andoni all contributed equally to the coding of the web platform.

\textbf{My personal contributions include}:
\begin{itemize}
    \item Supervision and testing of both projects
    \item Coding parts of the annotation software: Decoding of videos and gaze-tracker integration
    \item Packaging of the annotation software for Linux platforms
    \item Coding parts of the web-platform: Frontend, backend, packaging
    \item Documentation of the web-platform
\end{itemize}

\section{Introduction}

In section \ref{sec:feat_method}, we introduce our experimental framework for the study of feature learning in the frame of our general problem of sparse point-wise annotation.
In particular, several baseline methods are introduced along the deep learning approach used in \cite{lejeune18}.
Extensive experiments are performed and results are given in section \ref{sec:feats_results}.


\endinput

%%% Local Variables:
%%% mode: latex
%%% TeX-master: "../../main"
%%% End:
