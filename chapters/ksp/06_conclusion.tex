\section{Conclusion}
\label{sec:conclusion}
In this chapter, we presented a framework that allows for pixel-wise segmentation of objects of interest to be generated from sparse sets of 2D locations in video and volumetric image data. In this context, we have provided a strategy that produces an object segmentation by formulating the task as a global multiple-object tracking problem and solving it using an efficient K-shortest paths algorithm. Using an object model estimated from the sparse set of locations, we iteratively refine our solution by progressively improving our object model. To do this effectively, we introduce the use of image-object specific (IOS) features for our purpose and which are generated from an autoencoder that leverages the 2D object locations as a soft prior.

We show in our experiments that our approach is capable of reliably segmenting complex objects of interest over a wide range of image sequences and 3D volumes. Unlike previous methods, our \KSP ~method does not assume that the object is of a given size or any information about the background is known. Yet by combining our multi-path tracker and our object model, we achieve superior segmentation results compared to a number of state-of-the-art methods. Beyond this, we show that our results are stable under a number of conditions including the specific nature of the provided 2D locations, even when these are collected at framerate using a low cost gaze tracker.

While we demonstrate in our experiments that generated segmentations could be used to train supervised machine learning segmentation methods without suffering too greatly, we show that the performance of our method does depend on the spatial coverage of the provided 2D locations. In the future, we will look to further exploit inter-frame consistency to refine segmentations, in particular to recover fine object details. In addition, so far we have assumed that only a single object is within the data volume. As such, we will look to overcome this limitation and investigate how transfer learning strategies could benefit both feature extraction and segmentation towards this end. Last, in our current set-up, the size of the superpixels used can negatively impact the segmentation produced. To limit the impact of this shortcoming, the use of strategies that refine or merge superpixels to produce more accurate final results will be investigated.


%%% Local Variables:
%%% mode: latex
%%% TeX-master: "../../main"
%%% End:
