\begin{abstract}
\label{sec:orgheadline1}
Recent machine learning strategies for segmentation tasks have shown great ability when trained on large pixel-wise annotated image datasets. It remains a major challenge however to aggregate such datasets, as the time and monetary cost associated with collecting extensive annotations is extremely high. This is particularly the case for generating precise pixel-wise annotations in video and volumetric image data. To this end, this work presents a novel framework to produce pixel-wise segmentations using minimal supervision. Our method relies on 2D point supervision, whereby a single 2D location within an object of interest is provided on each image of the data. Our method then estimates the object appearance in a semi-supervised fashion by learning object-image-specific features and by using these in a semi-supervised learning framework. Our object model is then used in a graph-based optimization problem that takes into account all provided locations and the image data in order to infer the complete pixel-wise segmentation. In practice, we solve this optimally as a tracking problem using a K-shortest path approach. Both the object model and segmentation are then refined iteratively to further improve the final segmentation. We show that by collecting 2D locations using a gaze tracker, our approach can provide state-of-the-art segmentations on a range of objects and image modalities (video and 3D volumes), and that these can then be used to train supervised machine learning classifiers. 
\end{abstract}
%%% Local Variables:
%%% mode: latex
%%% TeX-master: "../../main"
%%% End:
