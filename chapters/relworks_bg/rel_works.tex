\section{Related Works}

Past research that address image and volume segmentation using sparse supervision can be roughly divided in two categories,
The first category, mainly focuses on computer vision techniques, while the second focuses on developping Machine Learning techniques.
In practice, both categories often overlap.

We now give an overview of the state-of-the-art in computer vision.
Next, we focus on \gls{ml} methods.


\subsection{Computer Vision Methods}
Another line of research focuses on the annotation protocol itself so as to complement the above efforts.
While the most tedious protocol requires that images be manually segmented at a pixel-level, many computer-vision approaches exist to facilitate this task.

Early contributions relied on the Active Contour model \cite{kass88}, which assumes that an initial contour is given and parameterized as the nodes of a spline curve, in this context called an elastic snake.
The segmentation problem is formulated as finding the deformation of the initial snake that minimizes an energy term.
In its simplest form, the energy term is composed of a term that determines the fitting of the snake to the edges of the image, while another term controls the smoothness of the snake.
The energy is then minimized using gradient descent.
Following works alleviate the problem of robustness brought by noisy edge maps by minimizing the Mumford-Shah functional \cite{chan01}, and parameterize the contour using the level-set method \cite{osher88}.

More recently, annotations in the form of scribbles were considered, where the annotator is asked to draw crude delineations of one or several objects of interest, along with a scribble on the background.
A first method that handles such kind of annotations relies on the Random-Walk algorithm \cite{grady06}.
The segmentation problem reduces to assigning to each unlabeled pixel a random-walker. The label of the latter pixel is then determined by the finding the label for which the walker has a maximum probability of reaching first.

Relying on the same annotation requirement, the graph-cut approach \cite{Boykov2006}
minimizes an Markov Random Field energy objective that considers unary terms (a scalar on each pixel), and a pairwise term that models the similarity of neighboring pixels.
Concretely, the image or volume is represented as a region adjacency graph where each pixel is assigned a node, and edges connect adjacent pixels.
Each annotated positive pixels is connected to a source node, while annotated negatives are connected to a sink node.
The energy minimization is then performed using the min-cut/max-flow algorithm \cite{goldberg88}.
Using the same algorithm, grab-cut \cite{rother04}, further simplifies the scribble-based annotation protocol by considering bounding-boxes around the object of interest, which provide crude labels on both foreground and background in a single stroke.

Along the same line, \cite{criminisi08}

\subsection{On the Wise Use of Annotation Budget}
The problem of scarcity of labeled data can also be posed in the following manner:
Given a limited annotation budget, i.e. when only a few hours of annotation work can be provided, how can one make the best use of it?

\gls{al} \cite{settles09} considers semi-supervised learning scenarios, where abundant unlabeled data are available.
\gls{al} algorithms select unlabeled samples that are the most informative to the \gls{ml} model, i.e. it assumes that many unlabeled samples represent redundant information to the late-stage \gls{ml} model.
In \cite{KonSznFua15}, authors devise a strategy to select unlabeled supervoxels of a 3D volume so as to segment volumes of various modalities.
The approach iteratively trains a classifier and takes its entropy as measure of uncertainty.
Combining the latter with a geometric prior, which takes into account intuitive rules on spatial coherence, they sample a batch of most uncertain samples.
The classifier is then re-trained using the newly annotated samples.

Crowd-sourcing, refers to the outsourcing of the annotation task to a high number of annotators \cite{orting19}.
Its use in the frame of medical imaging poses several limitations.
First, the difficulty of the annotation task can prohibit the effective use of crowd-sourcing, e.g when the structure of interest is hard to distinguish \cite{orting19}.
Another limitation is the nature of the data, i.e. one usually prefers to pre-classify brain scans and submit to workers the slices where tumors are present.
Last, one must usually integrate a test-task to ensure the competence of workers, or filter-out annotations provided by ``poorly performing'' workers \cite{park18}.

\subsection{Transfer Learning, Semi-supervised Learning, and Domain Adaptation}
\gls{tl} consists in using a \gls{ml} model trained for the task of interest, but using a training set of different nature than the targeted domain, e.g. train a \gls{cnn} to segment natural objects (trees, cars, ...) for the segmentation of surgical tools.
As pointed out by \cite{oliver18}, an important and implicit assumption in that context is that both datasets share similar data distributions.

\gls{ssl} is a broad category of \gls{ml} algorithm that combines traditional supervised learning tasks, where a fully labeled training dataset is available, along with an abundant unlabeled dataset.
The latter can, under specific conditions, boost the performance of an \gls{ml} model trained with labeled dataset only, i.e. in a fully-supervised manner.
The basic idea is that the ML model may acquire knowledge of the data distribution from the unlabeled set.
Here again, one assumes that both data domains share similar distributions.
In \cite{ghafoorian17}, authors show experimentally that a \gls{nn} trained to segment tumors on MRI brain scans acquired in a given MRI protocol, perform poorly when transfered to another protocol, as the shape, intensities, and contrast level are inconsistent.

\gls{da} addresses the problem of adapting the model so as to take into account domain shift, i.e. the discrepancies between the training and target domain.
In \cite{perone19}, authors devise a strategy to segment MRI scans of spinal cords using a model trained with images taken at a specific field-of-view, and transfer to another field-of-view in an unsupervised fashion.
Their approach leverages data-augmentation to enforce a consistency between the augmented and pristine images of the unlabeled set, while jointly training with the labeled set with traditional cross-entropy loss.
On the same image modality, \cite{li20} resort to adversarial training. They jointly train a segmentation model and a discriminator wit shared weights. The discriminator is trained to discriminate between the source and target domain.

%%% Local Variables:
%%% mode: latex
%%% TeX-master: "../../main"
%%% End:
